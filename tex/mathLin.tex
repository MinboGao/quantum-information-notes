\chapter{Mathmetical Preliminary: Linear algebra}

In the first few chapters, we will discuss some basic ideas in linear algebra, functional analysis and optimization theory that are useful in the study of quantum information and quantum computation. We will use the viewpoint of quantum computer scientist, so some notations may seem strange at first glance.

First, we introduce some basic notions in linear algebra using Dirac's symbol. This is the starting point of the quantum mechanics.

\section{Overview of Linear Algebra}

So here we face two important questions: what is the main subject of linear algebra? And why we need linear algebra in quantum information theory. 

For the first quesion, we have a quick answer from some textbooks: linear algebra usually focus on the study of vector spaces (in most cases we assume the dimension is finite) and linear transformations that act on them. So we need to just understand the notions of these two objects, and the methods that we adopt in analysing their properties, since some of them are useful in a broader sense, like in the study of quantum channels.

For the second question, I can't give a very "correct" answer. But I will try to explain what I think for this question: ideally, we need serious functional analysis for the general study of quantum mechanics. However, for simplicity, quantum computation just consider the easiest case: the qubits. In this case, linear algebra are good enough for us to understand the theory.

Now we enumerate some topics that we will introduce in the following sections:

\begin{itemize}
    \item Vector spaces
    \item Linear transformations
    \item Basis and matrices
    \item Tensor products
    \item Quadratic forms
    \item Canonical forms and diagonalization
    \item Useful matrix inequalities
\end{itemize}

\section{Vector spaces}

Vertor space can be seen as a module over a field.